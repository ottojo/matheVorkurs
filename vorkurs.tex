\documentclass{article}

\usepackage[utf8]{inputenc}
\usepackage[german]{babel}
\usepackage{enumitem}
\usepackage{csquotes}
\usepackage{mathtools}
\usepackage{amsfonts}
\usepackage{amsthm}

\title{Mathe Vorkurs 2017}
\author{Jonas Otto}
\date{September 2017}

\begin{document}

\maketitle
\tableofcontents

\section{Elementare Logik}
  \subsection{}
    \subsubsection{Definition}
      Eine Aussage ist ein Satz, der wahr oder falsch sein kann. Zwei Aussagen $A,B$
      heißen logisch äquivalent, in Zeichen $A \iff B$, falls sie den selben
      Wahrheitswert haben.

    \subsubsection{Bemerkung}
      \begin{enumerate}[label=(\roman*)]
        \item Es muss nicht bekannt sein, ob eine Aussage wahr oder falsch ist.
        \item Wahre Aussagen müssen nicht \enquote{nützlich} sein und umgekehrt.
        \item Äquivalente Aussagen müssen \enquote{nichts miteinander zu tun haben}.
      \end{enumerate}

    \subsubsection{Beispiele}
      \begin{enumerate}[label=(\roman*)]
        \item $A = ``\text{Es regnet}"$ (Wahr)
        \item $B = ``x+2"$ ist keine Aussage
        \item $C = ``2+3=5"$ (Wahr)
        \item $D = ``\text{Jede gerade Zahl außer 2 ist die Summe von zwei
        Prizahlen}"$ (Unbekannt)
        \item $E = ``\pi = \frac{333}{106}"$ falsch, aber nützlich
        \item $F = ``\text{Meine Hose ist blau}"$ und
        $G = ``\text{Ich trinke Kaffee}"$ sind logisch äquivalent
      \end{enumerate}

    In diesem Abschnitt seien $A,B,C$ stets Aussagen.
    \subsubsection{Definition (Junktoren und Wahrheitstabellen)}
      \begin{enumerate}[label=(\roman*)]
        \item $A \wedge B$ \enquote{$A$ und $B$}

          $A \wedge B$ ist wahr, wenn beide Aussagen $A,B$ wahr sind, sonst ist sie
          falsch.

          \begin{center}
            \begin{tabular}{ c c c }
              $A$ & $B$ & $A \wedge B$ \\ \hline
              f & f & f \\
              f & w & f \\
              w & f & f \\
              w & w & w \\
              \hline
            \end{tabular}
          \end{center}
        \item $A \vee B$ \enquote{$A$ oder $B$}
        \item $\neg A$ \enquote{nicht $A$}
        \item $A \implies B$ \enquote{$A$ impliziert $B$}
          \begin{center}
            \begin{tabular}{ c c c }
              $A$ & $B$ & $A \implies B$ \\ \hline
              f & f & w \\
              f & w & w \\
              w & f & f \\
              w & w & w \\
              \hline
            \end{tabular}
          \end{center}

          Man sagt, die Aussage $A$ ist hinreichend für $B$ und die Aussage $B$
          ist notwendig für $A$.
        \item $A \iff B$ \enquote{$A$ ist äquivalent zu $B$}
          \begin{center}
            \begin{tabular}{ c c c }
              $A$ & $B$ & $A \iff B$ \\ \hline
              f & f & w \\
              f & w & f \\
              w & f & f \\
              w & w & w \\
              \hline
            \end{tabular}
          \end{center}

      \end{enumerate}

      \subsubsection{Beispiel}

    \subsubsection{Satz (De Morgan)}
      \begin{enumerate}[label=(\roman*)]
        \item $\neg (A \wedge B) \iff (\neg A) \vee (\neg B)$
        \item $\neg (A \vee B) \iff (\neg A) \wedge (\neg B)$
      \end{enumerate}

    \subsubsection{Lemma (Hilfssatz)}
      \begin{enumerate}[label=(\roman*)]
        \item $\neg (\neg A) \iff A$
        \item $A \vee B \iff B \vee A$

          $A \wedge B \iff B \wedge A$
        \item $A \vee (B \vee C) \iff (A \vee B) \vee C$

          $A \wedge (B \wedge C) \iff (A \wedge B) \wedge C$
        \item $(A \iff B) \iff (A \implies B) \wedge (B \implies A)$
      \end{enumerate}

    \subsubsection{Satz (Kontraposition)}
      $\left( A \implies B \right) \iff \left( \neg B \implies \neg A \right)$

  \subsection{Naive Mengenlehre}
    \subsubsection{Definition (Cantor 1845-1918)}
      Eine Menge ist eine Zusammenfassung von wohlunterscheidbaren Objekten
      unserer Anschauung oder unseres Denkens zu einem Ganzen.

    \subsubsection{Bemerkung}
      \begin{enumerate}[label=(\roman*)]
        \item Naiver Mengenbegriff
        \item \enquote{Russelsche Antinomie} Naiver Mengenbegriff führt zu Widersprüchen!
        \item Bei der Bildung von \enquote{Mengen aller Mengen} muss man vorsichtig sein!
      \end{enumerate}

\section{Reelle Zahlen}
  \subsection{Der Körper der reellen Zahlen}
    \subsubsection{Definition}
      Ein Körper ist eine Menge $K$ mit zwei Verknüpfungen $+, \cdot$, die aus
      zwei Elementen $a,b \in K$ ein neues Element
      $(a+b) \in K, (a \cdot b) \in K$ zuordnen und dabei die folgenden
      Rechenregeln erfüllen.

      \begin{enumerate}[label=(A\arabic*)]
        \item Assoziativität von $+$

          $\forall a,b,c \in K: (a+b)+c = a+(b+c)$ kurz $a+b+c$
        \item Kommutativität

          $\forall a,b \in K: a+b=b+a$

        \item Es gibt ein Element $0 \in K, \forall a \in K: a+0 = 0+a = a$
      \end{enumerate}
      //Todo

\section{Beweismethoden}
  \subsection{Direkter Beweis}
    Wir wollen $A \implies B$ zeigen:\\
    Man nimmt an, dass $A$ wahr ist, und seien $A_n,\cdots,A_m$
    weitere Aussagen. Dann:\\
    $A \implies A_1 \implies A_2 \implies \dots \implies A_n \implies B$

    \subsubsection{Beispiel}
      Sei $n \in \mathbb{N}$ gerade. Dann ist auch $n^2$ gerade.

      \begin{proof} \mbox{} \\
        $A = ``n \text{ ist gerade}"$ \\
        $B = ``n^2 \text{ ist gerade}"$ \\
        $\text{Wir wollen } A \implies B \text{ zeigen.}$
        \begin{align*}
          A \implies n=2k, k \in \mathbb{N}
          \implies n^2=(2k)^2 = 4k^2 =
          2\cdot\underbrace{(2k^2)}_{\in \mathbb{N}}
        \end{align*}
        $n^2 \text{ ist gerade.}$
      \end{proof}

    \subsubsection{Beispiel}
      Wir betrachten ein $8x8$ Schachfeld, bei dem zwei gegenüberliegende Felder
      entfernt wurden. Wir legen Dominosteine auf das Feld, die jeweils zwei
      Felder überdecken. \\
      Behauptung: Das ganze Feld kann nicht komplett bedeckt werden.

      \begin{enumerate}[label=\arabic*.]
        \item Lösungsmöglichkeit: Probiere alle Möglichkeiten durch.\\
          Beweis per Fallunterscheidung
        \item Lösungsmöglichkeit: Direkter Beweis:
          \begin{proof} \mbox{}
            \begin{enumerate}[label=(\roman*)]
              \item Das Feld hat 64 Felder, davon sind 32 weiß und 32 schwarz.
                Nach dem Entfernen sind noch 62 Felder übrig.
              \item Die entfernten Felder haben beide die Farbe schwarz. Also
                gibt es noch 32 weiße und 30 schwarze Felder.
              \item Ein Dominostein bedeckt immer 2 benachbarte Felder. Diese
                haben unterschiedliche Farben.
              \item Auf das neue Feld passen maximal 31 Steine.
              \item Nach 30 Steinen bleiben 2 Felder übrig. Nach (iii)  bleiben
                noch zwei weiße Felder übrig. Der letzte Stein müsste dann zwei
                weiße Felder bedecken. Nach (iii) ist das unmöglich.
            \end{enumerate}
            Also gibt es keine solche Bedeckung.
          \end{proof}
      \end{enumerate}

  \subsection{Indirekter Beweis}
    Wir wollen $A \implies B$ zeigen. Nach dem Prinzip der Kontraposition ist
    dies äquivalent zu $\neg A \implies \neg B$

    \subsubsection{Beispiel}
      Sei $n \in \mathbb{N}$. Wenn $5$ die Zahl $n$ nicht teilt, dann teilt auch
      $10$ die Zahl $n$ nicht.
      \begin{proof} \mbox{} \\
        $A = ``5$ teilt $n$ nicht" \\
        $B = ``10$ teilt $n$ nicht"
        \begin{align*}
          A &\implies B \\
          \neg B &\implies \neg A
        \end{align*}
        $\neg A = `` 5$ teilt $n$" \\
        $\neg B = `` 10$ teilt $n$" \\
        Wir nehmen an, dass $\neg B$ wahr ist. Jetzt müssen wir $\neg A$ zeigen. \\
        Da $\neg B$ ist wahr, gilt, folgt $\exists k \in \mathbb{N}$
        mit $n=10k = 5 \cdot \underbrace{(2k)}_{\in \mathbb{N}}$ \\
        $\implies ``5 \text{ teilt } n$" $ = \neg A$
      \end{proof}

  \subsection{Beweis per Widerspruch}
    Wir wollen $A \implies B$ zeigen. Wir nehmen an, dass $A$ gilt, aber dass auch
    $\neg B$ gilt und führen das zu einem Widerspruch. Also muss $\neg B$ falsch
    sein, also ist $B$ wahr.

    \subsubsection{Satz (Euklid ~200 v.Chr.)}
      Es gibt unendlich viele Primzahlen.
      \begin{proof}
        Wir nehmen an, es gibt nur endlich viele Primzahlen, etwa \\
        $p_1, p_2, \dotsc, p_n \in \mathbb{N}$. Wir führen dies zu einem Widerspruch. \\
        Definiere die Zahl $q = p_1 \cdot p_2 \dotsm p_n +1$.\\
        Beachte: Keine der Zahlen $p_1 \dotso p_n$ teilt 1 (es bleibt immer ein
        Rest $1$!). Dies ist ein Widerspruch zu 3.3.2. \\
        Also gibt es unendlich viele Primzahlen.
      \end{proof}

    \subsubsection{Lemma}
      Sei $n \in \mathbb{N}, n>1$, dann ist $n$ durch eine Primzahl teilbar
      \begin{proof}
        //Later
      \end{proof}

    \subsubsection{Bemerkung}
      \begin{enumerate}[label=(\roman*)]
        \item Ein Paar Primzahlen mit Abstand 2 heißt Primzahl-Zwilling.\\
          Etwa: $(3,5), (5,7), (11,13), \dotsc$ \\
          Es ist unbekannt ob es unendlich viele Primzahlzwillinge gibt.
        \item Yitang Zhang bewies 14. Mai 2015: Es gibt unendlich viele Paare von
          Primzahlen mit Abstand $70.000.000$.
        \item $\vdots$
        \item April 2014: Abstand $246$
      \end{enumerate}

  \subsection{Vollständige Induktion}
    Dies ist eine Methode um eine Aussage über alle natürlichen Zahlen zu beweisen.

    \subsubsection{Satz}
      Sei für $n \in \mathbb{N} A(n)$ eine Aussage.
      \begin{enumerate}[label=(\roman*)]
        \item Es gelte $A(1)$ ist wahr. (Induktionsanfang)
        \item Es gelte $``A(n) \implies A(n+1)"$ (Induktionsschritt) für
          $n \in \mathbb{N}$ fest, aber beliebig.
      \end{enumerate}

      Dann gilt $A(n)$ für alle $n \in \mathbb{N}$

    \subsubsection{Beispiel \enquote{Kleiner Gauß}}
      Für alle $n \in \mathbb{N}$ gilt $\displaystyle\sum_{k=1}^{n} k = \frac{n(n+1)}{2}$
      \begin{proof} \mbox{} \\
        IA: Sei $n = 1$ \\
        $\displaystyle\sum_{k=1}^{1} k = 1 = \frac{1 \cdot (1+1)}{2} \implies A(1) \text{ ist wahr.}$ \\
        IS: Sei $n \in \mathbb{N}$ beliebig, aber fest. \\
        Angenommen $A(n)$ ist wahr (Induktionshypothese \enquote{IH}) \\
        Wir zeigen, dass unter dieser Annahme auch $A(n+1)$ wahr ist.
        \begin{align*}
          \displaystyle\sum_{k=1}^{n+1} k = \displaystyle\sum_{k=1}^{n} k + n+1
          &= \frac{n \cdot (n+1)}{2} + n+1 \\
          &= \frac{n \cdot (n+1)}{2} + \frac{2 \cdot (n+1)}{2} \\
          &=\frac{(n+1) \cdot (n+2)}{2}
        \end{align*}
      \end{proof}

    \subsubsection{Zusammenfassung}
      \begin{itemize}
        \item [IA:] Die Aussage gilt für $n=1$
        \item [IH:] Wir nehmen an, dass die Aussage für ein festes, aber
          beliebiges $n \in \mathbb{N}$ gilt
        \item [IS:] Zeige unter Verwendung von IH dass $A$ für $n+1$ gilt
      \end{itemize}

    \subsubsection{Beispiel}
      //Todo

    \subsubsection{Beispiel}
      //Todo

    \subsubsection{Beispiel Bernoulli-Ungelichung}
      //Todo

    \subsubsection{Bemerkung}
      Der Induktionsanfang muss nocht unbedingt bei $n=1$ beginnen.

    \subsubsection{Beispiel}
      //Todo

\section{Funktionen}
  \subsection{}
    \subsubsection{Definition}
      Seien $X,Y$ Mengen. \\
      Eine Funktion $f$ ist eine Vorschrift, die jedem $x \in X$ ein eindeutiges
      $y \in Y$ zuordnet. Das $x \in X$ zugeordnete $y \in Y$ wird mit
      $y=f(x)$ bezeichnet. \\
      Notation:
      \begin{align*}
        f: X &\to Y \\
        X \ni x &\mapsto f(x)
      \end{align*}
      Die Menge $X$ heißt Definitionsmenge/-bereich, die Menge $Y$ heißt
      Zielmenge. Das $x \in X$ zugeordnete $y \in Y, y=f(x)$ heißt Bild von $x$
      unter $f$. Das $x \in X$ mit $f(x)=y$ heißt ein Urbild von $y$ unter $f$.

      I'm not going to draw such a fancy diagram like on the blackboard using
      \LaTeX btw, i'm not THAT crazy!

    \subsubsection{Beispiele}
      \begin{enumerate}[label=(\roman*)]
        \item $f: \mathbb{R} \to \mathbb{R}, x \mapsto 2x+1$ \\
          $f(x)=2x+1$
        \item $f: \mathbb{R} \to \mathbb{R}, x \mapsto x^2$ \\
          $f(x)=x^2$
        \item $f: \mathbb{R} \to [0,\infty), x \mapsto x^2$ \\
          Hier hat jedes $y \in Y=[0,\infty)$ zwei Urbilder, nämlich
          $x_1 = \sqrt{y}, x_2 = -\sqrt{y}$
        \item $f: [0,\infty) \to [0,\infty), x \mapsto x^2$, dann hat jedes
          $y \in [0,\infty)$ genau ein Urbild $x \in [0,\infty)$
      \end{enumerate}

    \subsubsection{Definiton}
      Sei $f: X \to Y$ eine Funktion und $X_1 < X, Y_1 < Y$, dann heißt
      \begin{enumerate}[label=(\roman*)]
        \item $f(X_1) := \{ y \in Y, y=f(x),x \in X_1\}$ das Bild von $X_1$ unter $f$
        \item $f^{-1}(Y_1) := \{ x \in X, f(x) \in Y_1 \}$ das Urbild von $Y_1$
        \item die Funktion $f$ \emph{surjektiv}, falls $f(X) = Y$, d.h. das Bild
          von $X$ ist die gesamte Menge $Y$ \\
          $f$ ist surjektiv $\iff \forall y \in Y, \exists x \in X: f(x)=y$
        \item die Funktion $f$ \emph{injektiv}, wenn \\
          $x_1, x_2 \in X, x_1 \neq x_2 \implies f(x_1) \neq f(x_2)$ \\
          $\iff f(x_1) = f(x_2) \implies x_1 = x_2$ \\
          $\iff$ Falls $y \in Y$ ein Urbild $x \in X$ hat, fann ist dies eindeutig
          bestimmt.
        \item die Funktion $f$ \emph{bijektiv}, falls sie sowohl injektiv als
          auch surjektiv ist, d.h. falls jedes Element $y \in Y$ genau ein Urbild
          $x \in X$ besitzt.
      \end{enumerate}

    \subsubsection{Beispiel}
      \begin{enumerate}[label=(\roman*)]
        \item $f: \mathbb{N} \to \mathbb{N}, x \mapsto 2x$
          \begin{align*}
            X_1&=\{2,3\} & Y_1&=\{8,14,100\} \\
            f(X_1) &= \{4,6\}, & f^{-1}(Y_1) &= \{4,7,50\}
          \end{align*}
        \item $f: \mathbb{R} \to \mathbb{R}, x \mapsto 2x+1$ ist bijektiv: \\
          \begin{enumerate}[label=\alph*)]
            \item $f$ ist surjektiv. Sei $y \in \mathbb{R}$ beliebig. \\
              Ziel: finde $x \in \mathbb{R}, f(x)&=y$ \\
              \begin{align*}
                f(x)&=y \\
                \iff 2x+1&=y \\
                \iff x&=\frac{y-1}{2}
              \end{align*}
              $f(x)=2(\frac{y-1}{2})+1&=y$
            \item $f$ ist injektiv. Seien $x_1, x_2 \in X$ mit
              \begin{align*}
                f(x_1)=f(x_2) &\iff 2x_1+1 = 2x_2+1 \\
                &\iff x_1=x_2 \\
                &\implies f \text{ ist injektiv}
              \end{align*}
          \end{enumerate}
      \end{enumerate}

\section{Differenzialrechnung}
  \subsection{Definition des Integrals}
    \subsubsection{Definition}
      Sei $I \subset \mathbb{R}$ ein offenes Intervall und $x_0 \in I$ und
      $c \in \mathbb{R}, f: I \to \mathbb{R}$
      \begin{enumerate}[label=(\roman*)]
        \item $\displaystyle\lim_{x \to \infty} f(x)=c$ //Todo (Unleserlich in Mitschrift)
        \item $f$ heißt stetig in $x_0$ falls $\displaystyle\lim_{x \to x_0} = f(x_0)$
        \item Wir sagen $f$ ist stetig, falls $f$ in jedem Punkt $x_0 \in I$ stetig ist.
        \item $f$ heißt differenzierbar in $x_0$ falls
          \begin{displaymath}
            \lim_{x \to x_0} \frac{f(x)-f(x_0)}{x-x_0} = f'(x_0)
          \end{displaymath}
          existiert. In dem Fall heißt $f'(x_0)$ die Ableitung von $f$ in $x_0$.
        \item $f$ heißt differenzierbar in $I$, falls $f$ in jedem Punkt $x_0 \in I$
          differenzierbar ist. In dem Fall heißt $f': I \to \mathb{R}, x_0 \mapsto f'(x_0)$
          die Ableitung von $f$.
        \item $f$ heißt stetig differenzierbar in $I$, falls $f$ differenzierbar
          ist und $f'$ stetig ist auf $I$.
      \end{enumerate}





\end{document}
